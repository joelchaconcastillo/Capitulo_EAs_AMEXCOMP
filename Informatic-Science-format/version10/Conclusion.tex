La convergencia prematura es una de las debilidades más importantes de los \EAS{}.
%
Una de las formas de lidiar con este problema consiste en inducir un balanceo adecuado entre exploración e intensificación.
%
Sin embargo, obtener este balanceo no es una tarea trivial debido a que cada problema de optimización tiene características distintas y por
ello han surgido numerosos esquemas para tratar esta problemática.
%
Con base en esto se han desarrollado varias taxonomías con el propósito de clasificar las diferentes formas en que se puede preservar y promover la diversidad.
%
En este trabajo, con el fin de ilustrar como se pueden realizar diseños de \EAS{} que tomen en cuenta el balanceo entre exploración e intensificación
se presentan dos ejemplos.
%
De forma general, las propuestas establecen un comportamiento dinámico de algún componente del \EA{} teniendo en cuenta para ello el criterio de paro 
y el instante en que se encuentra la ejecución.
%
Así, se obtuvo un balanceo adecuado en el que en las primeras fases se promueve más exploración y en las últimas se promueve más la intensificación. 
%
De modo más específico, en la primera propuesta se modifica evolución diferencial por medio de una nueva fase de reemplazo con el propósito de conseguir este balanceo. 
%
Además, esta contribución incorpora una población élite con el propósito de proporcionar soluciones de calidad tanto a mediano como a largo plazo.
%
Con base en la validación experimental llevada a cabo, se observa que esta propuesta mejora a los mejores algoritmos que se habían propuesto en una competición
que consideró las funciones de prueba usadas en este capítulo.
%
Posteriormente, se presenta un análisis del operador de cruza \SBX{} y se desarrolló una variante que modifica varios componentes de forma dinámica
considerando también el criterio de paro y las generaciones transcurridas. 
%
Con base en los resultados obtenidos con problemas muy populares del ámbito de optimización multi-objetivo, se muestra que usar el operador \SBX{} modificado ofrece ventajas muy importantes
en varios \MOEAS{} y para diferentes indicadores.
%
De forma general se observa que obtener un balanceo apropiado entre exploración e intensificación es realmente importante, y que
esto se puede conseguir utilizando técnicas radicalmente diferentes aunque siguiendo los mismos principios de diseño.

El campo de diseño de \EAS{} es un campo muy activo y en el que hay muchas ideas aún por explorar.
%
Uno de los aspectos importantes es que las estrategias de control de diversidad ofrecen muchas ventajas a largo plazo, 
pero se debe hacer el esfuerzo por conseguir que se pueda reducir el número de evaluaciones requeridas para obtener 
resultados prometedores.
%
Otro aspecto importante es que, en general, la mayor parte de las estrategias de control de diversidad introduce nuevos parámetros, por lo que es importante
combinar estos mecanismos con estrategias de control de parámetros adaptativas o auto-adaptativas para facilitar su utilización.
%
Finalmente, para el caso específico del \DSBX{}, sería interesante integrarlo con técnicas de aprendizaje para lidiar con problemas con dependencias y mejorar
así aún más el rendimiento de la propuesta.
