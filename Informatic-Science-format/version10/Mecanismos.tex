La convergencia prematura es una problemática muy conocida en el ámbito de los \EAS{} por lo que se han desarrollado gran cantidad de técnicas 
para lidiar con la misma~\cite{pandey2014comparative}.
%
Estas técnicas modifican de manera directa o indirecta la cantidad de diversidad mantenida por el algoritmo~\cite{Joel:Crepinsek}
y varían desde técnicas generales hasta mecanismos dependientes de un problema dado.
%
En este apartado se revisan algunas de las técnicas generales más populares.
%
Inicialmente, se describen algunas maneras de clasificar a este tipo de estrategias, para posteriormente
describir mecanismos clásicos específicos, así como algunas estrategias más novedosas que se basan en modificar
la estrategia de reemplazamiento.
%
Finalmente, dado que en este capítulo se extiende \DE{}, se revisan también los trabajos de esta área que tienen una relación
significativa con el manejo de diversidad.
%
Se invita a los lectores que requieran conocer estos mecanismos con más detalle a consultar~\cite{Joel:Crepinsek} así como los artículos
específicos en que cada método es propuesto.

\subsection{Clasificaciones de mecanismo para promover la diversidad}

Debido a la gran cantidad de métodos desarrollados en esta área, se han propuesto varias clasificaciones de los mismos.
%
Liu et al.\cite{liu2009explore} propuso diferenciar entre los enfoques uni-proceso y multi-proceso.
%
En el enfoque uni-proceso se modifica la preservación de la diversidad actuando sobre un único componente del \EA{}. 
%
Es importante notar que en los enfoques uni-proceso no se excluye el uso de otros componentes en el proceso de exploración y/o intensificación, 
sino que más bien, si no se consigue el balanceo adecuado, sólo se modifica una componente hasta conseguir el comportamiento deseado.
%
Por otra parte, en los enfoques multi-proceso se tienen en cuenta las implicaciones de varios componentes en el balanceo, y se actúa modificando o rediseñando varios de ellos
hasta conseguir el comportamiento adecuado.
%
Los esquemas uni-proceso son mucho más habituales actualmente~\cite{Crepinsek:13}, y en particular las dos propuestas incluidas en este capítulo son mecanismos
uni-proceso.

Extendiendo a la anterior, se propuso una clasificación más específica~\cite{Crepinsek:13}, en la que se tiene en cuenta cuál es la componente 
que se cambia para categorizar a los métodos.
%
En este sentido, los más populares son los siguientes:

\begin{itemize}
\item \textbf{Enfoques basados en la selección}: son los más clásicos y se basan en cambiar la presión de selección que se produce hacia las zonas promisorias a la hora de realizar
la selección de padres.
\item \textbf{Enfoques basados en población}: se modifica el modelo poblacional, utilizando algunas técnicas como hacer variar el tamaño de la población de forma
dinámica, eliminar individuos duplicados, utilizar técnicas de infusión o estblecer un mecanismo de islas con migraciones.
\item \textbf{Enfoques basados en el cruce y/o mutación}: se basan en rediseñar los operadores de cruce y/o mutación habitualmente teniendo en cuenta el problema específico
que se está resolviendo, en aplicar restricciones sobre el emparejamiento o en incluir operadores disruptivos que podrían ser utilizado sólo en ciertos instantes del proceso
de optimización.
\end{itemize}


\subsection{Esquemas clásicos para administrar la diversidad}

Los primeros \EAS{} se basaron principalmente en esquemas generacionales que no incluían fase de reemplazo.
%
En estos esquemas la selección de padres era la principal responsable de que se muestreara con mayor probabilidad las zonas
más promisorias encontradas hasta el momento, y por tanto muchos de los primeros esquemas que trataron de evitar la convergencia 
prematura se basaron en modificar el proceso de selección de padres.
%
Así, en los 90s se desarrollaron varios esquemas que alteraban la presión de selección~\cite{eiben2003introduction}
de forma estática o dinámica.
%
Sin embargo, en base a varios estudios teóricos y experimentales se observó que actuar exclusivamente sobre los operadores de selección no es suficiente, especialmente
cuando se quieren realizar ejecuciones a largo plazo, ya que se requerirían poblaciones excesivamente grandes para mantener un grado adecuado de diversidad.

Otra alternativa fue modificar los modelos poblaciones, encontrando en este grupo los esquemas basados en islas~\cite{alba2005parallel}, los celulares
o más recientemente los basados en clústeres~\cite{gao2014cluster}.
%
La idea de introducir restricciones en el emparejamiento, principalmente en base a la ubicación de los individuos en el espacio de búsqueda también
ha sido bastante exitosa aunque controlar los mismos para obtener el balanceo apropiado ante diferentes criterios de parada
es bastante complejo.
%
En algunos casos resultó ser más prometedor promover el emparejamiento entre individuos no similares~\cite{Joel:CHC}, mientras que en otros escenarios 
se hace exactamente lo opuesto~\cite{deb1989investigation}.
%
Otro problema común de muchas de las estrategias anteriores es que suelen introducir parámetros adicionales, por lo que el proceso de ajuste de parámetros, que ya de por sí
es un problema importante en los \EAS{}, se vuelve aún más complejo.
%
Es importante resaltar que todas estas estrategias clásicas no evitan por completo la convergencia sino que la idea es disponer de mecanismos para acelerarla o retrasarla.

Otra alternativa diferente ha sido adaptar la fase de variación.
%
En este sentido se han desarrollado diversas técnicas para controlar los parámetros que se consideran en la variación con el propósito de 
adaptar el balanceo entre exploración e intensificación.
%
En algunos casos esto se consigue usando distintos valores en los parámetros para distintas etapas a lo largo del proceso de optimización~\cite{yu2014differential},
mientras que en otros casos se hacen cambios más drásticos y se consideran varios operadores con distintas propiedades~\cite{lobo2007parameter}.
%
También existen mecanismos adaptativos que usan una memoria para almacenar información histórica sobre los efectos de la variación
y en base a ello irla modificando~\cite{yuen2009genetic}.
%
Cabe destacar que en la mayor parte de estos esquemas no se considera la diversidad de forma directa, sino que sólo se considera para analizar el comportamiento
y en base a ello rediseñar.

Finalmente, un esquema muy sencillo pero no por ello menos importante es el basado en reinicios.
%
En estos esquemas, en lugar de evitar la convergencia acelerada, se aplica un reinicio total o parcial de la población cada cierto número de
generaciones o cuando se detecta que la población ha convergido.
%
En base a esto se han propuesto diversas estrategias para establecer los puntos de reinicio~\cite{jansen2002analysis}.
%
Estos esquemas se implementan de forma muy sencilla y en algunos casos han proporcionado mejoras significativas~\cite{koumousis2006saw} por lo que
es un método a tener en cuenta, al menos como alternativa inicial.
%
Es común comibnar las estrategias basadas en reinicio con alguna de las técnicas anteriores, ya que las anteriores están basadas en mantener
la diversidad, mientras que en esta última el objetivo es recuperar la diversidad.

\subsection{Esquemas de reemplazamiento basados en diversidad}

Recientemente se han propuesto diversos de mecanismos que modifican la fase de reemplazo para preservar la diversidad.
%
La idea principal de estos esquemas es inducir un grado de exploración adecuado diversificando a los individuos sobrevivientes,
de forma que los operadores de reproducción puedan generar nuevas soluciones en diferentes regiones en las siguientes generaciones.
%
El principal principio en que se sustentan estos métodos es que los operadores de cruce tienen efecto de exploración al 
considerar individuos distantes e intensificación al considerar individuos próximos~\cite{eiben1998evolutionary}.

El esquema de pre-selección propuesto por Cavicchio~\cite{grefenstette1986optimization} es uno de los primeros estudios que utilizan 
la fase de reemplazamiento para controlar la diversidad.
%
El esquema inicial de Cavicchio se extendió para generar el \textit{amontonamiento o crowding}~\cite{de1975analysis}, el cual ha sido muy popular 
en los últimos años~\cite{mahfoud1992crowding, mengshoel2014adaptive}.
%
El principio del crowding se basa en que los nuevos individuos que entren en la población sustituyan a individuos similares de generaciones anteriores, y en base
a este principio, se han formulado diversas implementaciones.


\subsection{Diversidad en evolución diferencial}
%\subsection{Diversity in Differential Evolution}

Los algoritmos basados en \DE{} son altamente susceptibles a la pérdida de diversidad debido a que se basan en una estrategia de selección muy elitista.
%
Debido a ello, se han desarrollado varios análisis para lidiar con este problema.
%
Dado que en el área se conoce al menos de manera general, las implicaciones que tiene cada parámetro sobre la diversidad, algunos autores han trabajado
en estimar de forma teórica cuáles deben ser los parámetros adecuados para que se produzca cierto tipo de comportamiento~\cite{zaharie2003control}.
%
Otros autores han estudiado el efecto que tiene la norma de los vectores diferencia sobre la mutación~\cite{montgomery2009differential} y en base
a ello se han propuesto mecanismos que prohiben ciertos movimientos que pueden resultar perjediciales~\cite{montgomery2012simple}.
%
En este último estudio, el tipo de movimientos aceptados varía a lo largo de la ejecución, descartando en concretos movimiento con norma
menores a un umbral que es decrementado conforme transcurren las generaciones.
%
Además, se han propuesto otras formas para establecer los movimientos aceptados~\cite{bolufe2013differential}.

Una alternativa distinta se basa en alterar el operador de selección~\cite{sa2008exploration}.
%
Específicamente, con el propósito de mantener mayor diversidad en la población se altera la presión utilizando una selección probabilística que
permite escapar en algunos casos de las bases de atracción de óptimos locales.
%
Sin embargo, este método no es demasiado robusto debido a que considera la aptitud para definir las probabilidades para seleccionar a un individuo por lo que ciertas 
transformaciones de la función pueden modificar de forma drástica el tipo de búsqueda que se realiza.

Finalmente, la variante de \DE{} con \textit{Diversidad de la Población Auto-Mejorado} (\textit{Auto-Enhanced Population Diversity} - \textsc{aepd}) 
mide la diversidad de forma explícita y cuando se detecta la existencia de un nivel bajo de diversidad en la población, se 
lanza un mecanismo de diversificación~\cite{yang2015differential}.
%
Este esquema se ha extendido para incluir diferentes esquemas de perturbación~\cite{zhao2016differential}.

Es interesante hacer notar que las variantes de \DE{} que alcanzaron los primeros lugares en varias competencias de optimización durante los últimos años
no consideran estas modificaciones, y además estas variantes no han sido incorporadas en las herramientas de optimización más populares.
%
Esto se puede debe a que muchos de los concursos están orientados a obtener resultados en un número de evaluaciones bastante limitado en lugar de a largo plazo,
que es el ámbito en el que más beneficios suelen dar los mecanismos de control de diversidad.


