La convergencia prematura en una problemática muy conocida en el ámbito de los \EAS{} por lo que se han desarrollado gran cantidad de técnicas 
para lidiar con la misma~\cite{pandey2014comparative}.
%
Estas técnicas modifican de manera directa o indirecta la cantidad de diversidad mantenida por el algoritmo~\cite{Joel:Crepinsek}
y varían desde técnicas generales hasta mecanismos completamente dependientes de un problema dado.
%
En este apartado se revisan algunas de las técnicas generales más populares.
%
Inicialmente, se describen algunas maneras de clasificar a este tipo de estrategias, para posteriormente
describir mecanismos clásicos específicos, así como algunas estrategias más novedosas que se basan en modificar
la estrategia de reemplazamiento.
%
Finalmente, dado que en este capítulo se extiende \DE{}, se revisan también los trabajos de esta área que tienen una relación
significativa con el manejo de diversidad.
%
Se invita a los lectores que requieran conocer estos mecanismos con más detalle a consultar~\cite{Joel:Crepinsek} así como los artículos
específicos en que cada método es propuesto.

\subsection{Clasificaciones de mecanismo para promover la diversidad}

Debido a la gran cantidad de métodos desarrollados en esta áre, se han propuesto varias clasificaciones de los mismos.
%
Liu et al.\cite{liu2009explore} propuso diferenciar entre los enfoques uni-proceso y multi-proceso.
%
En el enfoque uni-proceso se modifica el balanceo entre exploración e intensificación considerando la modificación de un único componente del \EA{}. 
%
Así, es habitual diseñar los operadores de cruce con el objetivo de que en las primeras etapas, cuando la población es aún diversa, promueva la exploración
y en las fases finales, cuando la población es menos diversa, promueva la intensificación.
%
Es importante notar que en los enfoques uni-proceso no se excluye el uso de otros componentes en el proceso de exploración y/o intensificación, 
sino que más bien, si no se consigue el balanceo adecuado, sólo se modifica un componente hasta conseguir el comportamiento deseado.
%
Por otra parte, en los enfoques multi-proceso se tienen en cuenta las implicaciones de varios componentes en el balanceo, y se actúa modificando o rediseñando varios de ellos
hasta conseguir el comportamiento adecuado.
%
Los esquemas uni-proceso son mucho más habituales actualmente~\cite{Crepinsek:13}, y en particular las dos propuestas incluidas en este capítulo son mecanismos
uni-proceso.

Extendiendo esto se propuso una clasificación más específica~\cite{Crepinsek:13}, en los que se tiene en cuenta cuál es la componente específica que se cambia.
%
En este sentido, los más populares son los siguientes:

\begin{itemize}
\item \textbf{Enfoques basados en población}: se modifica el modelo habitual poblacional, utilizando algunas técnicas como hacer variar el tamaño de la población de forma
dinámica, eliminar individuos duplicados, utilizar técnicas de infusión o realizar migraciones entre poblaciones.
%
\item \textbf{Enfoques basados en la selección}: son los más clásicos y se basan principalmente en cambiar la presión de selección hacia zonas promisorias a la hora de realizar
la selección de padres.
%
\item \textbf{Enfoques basados en el cruce y/o mutación}: se basan en rediseñar los operadores de cruce y/o mutación habitualmente teniendo en cuenta el problema específico
que se está resolviendo, en aplicar restricciones sobre el emparejamiento o en incluir operadores disruptivos que podrían ser utilizado sólo en ciertos instantes del proceso
de optimización.
\end{itemize}


\subsection{Esquemas clásicos para administrar la diversidad}

Los primeros \EAS{} se basaron principalmente en esquemas generacionales que no incluían fase de reemplazo, y por tanto
muchos de los primeros esquemas que trataron de evitar la convergencia prematura se basaron en modificar el proceso de selección
de padres.
%
Así, en los 90s se desarrollaron varios esquemas que alteraban la presión de selección~\cite{eiben2003introduction}
de forma estática o dinámica.
%
Sin embargo, en base a varios estudios teóricos y experimentales se observó que actuar exclusivamente sobre los operadores de selección no es suficiente, especialmente
cuando se quieren realizar ejecuciones a largo plazo, ya que se requerirían poblaciones excesivamente grandes.

Otra alternativa fue modificar los modelos poblaciones, encontrando en este grupo los esquemas basados en islas~\cite{alba2005parallel}, los celulares
o más recientemente los basados en clústeres~\cite{gao2014cluster}.
%
La idea de introducir restricciones en el emparejamiento, principalmente en base a la ubicación de los individuos en el espacio de búsqueda también
ha sido bastante exitosa aunque controlar los mismos para obtener el balanceo apropiado ante diferentes criterios de parada
es un aspecto bastante complejo.
%
De hecho, en algunos casos parece ser más prometedor promover el emparejamiento entre individuos no similares~\cite{Joel:CHC}, mientras que en otros escenarios 
se hace exactamente lo opuesto~\cite{deb1989investigation}.
%
Otro problema común de muchas de las estrategias anteriores es que suelen introducir parámetros adicionales, por lo que el proceso de ajuste de parámetros, que ya de por sí
es un problema importante de los \EAS{} se vuelve aún más complejo.
%
Es importante resaltar que todas estas estrategias clásicas no evitan por completo la convergencia sino que su efecto es retrasarla.
%
Por lo tanto estas estrategias se podrían introducir con mecanismos adicionales y varias de ellas podrían usarse de forma simultánea.

Otra alternativa diferente ha sido adaptar la fase de variación.
%
En este sentido se han desarrollado diversas técnicas para controlar los parámetros que se consideran en la variación con el propósito de 
adaptar el balance entre exploración e intensificación.
%
En algunos casos esto se consigue usando distintos valores en los parámetros para distintas etapas a lo largo del proceso de optimización~\cite{yu2014differential},
mientras que en otros casos se hacen cambios más drásticos y se consideran varios operadores con distintas propiedades~\cite{lobo2007parameter}.
%
También existen mecanismos adaptativos que usan una memoria para almacenar información histórica sobre los efectos de la variación
y en base a ello irla modificando~\cite{yuen2009genetic}.
%
Cabe destacar que en la mayor parte de estos esquemas no se considera la diversidad de forma directa, sino que sólo se considera para analizar el comportamiento
y en base a ello rediseñar.

Finalmente, un esquema muy sencillo pero no por ello menos importante es el basado en reinicios.
%
En estos esquemas, en lugar de evitar la convergencia acelerada, se aplica un reinicio total o parcial de la población cada cierto número de
generaciones o cuando se detecta que la población ha convergido.
%
En base a esto se han propuesto diversas estrategias para establecer los puntos de reinicio~\cite{jansen2002analysis}.
%
Estos esquemas se implementan de forma muy sencilla y en algunos casos han proporcionado mejoras significativas~\cite{koumousis2006saw} por lo que
es un método a tener en cuenta, al menos como alternativa inicial.
%
Es común comibnar las estrategias basadas en reinicio con alguna de las técnicas anterior, ya que las anteriores están basadas en mantener
la diversidad, mientras que en esta el objetivo es recuperar la diversidad.

\subsection{Esquemas de reemplazamiento basados en diversidad}

Este tipo de mecanismos modifican la fase de reemplazo para preservar la diversidad.
%
La idea principal de estos esquemas es inducir un grado de exploración adecuado diversificando a los individuos sobrevivientes.
%
Esto se sucede ya son exploradas más regiones del espacio de búsqueda si se mantienen una población cuya diversidad sea elevada.
%
Además, los operadores de cruce tienen un efecto de exploración al considerar individuos distantes~\cite{eiben1998evolutionary}.
%
Particularmente, el esquema de pre-selección propuesto por Cavicchio's~\cite{grefenstette1986optimization} es uno de los primeros estudios que utilizan la fase de reemplazamiento para controlar la diversidad.
%
Posteriormente, esta pre-selección se extendió para generar el \textit{amontonamiento} (crowding) \cite{de1975analysis} el cual ha sido muy popular en los últimos años~\cite{mahfoud1992crowding, mengshoel2014adaptive}.
%
El principio del crowing se basa en forzar el ingreso de nuevos individuos los cuales únicamente sustituyen a sus padres con los que son similares.

La principal razón, es que en esa época la mayoría de esquemas eran generacionales, por lo tanto la presión de selección estaba definida principalmente en la selección de padres.
%
Sin embargo, se descubrió que únicamente considerando la selección de padres no es suficiente para aliviar la convergencia prematura~\cite{blickle1996comparison}.
%
Posteriormente, una gran cantidad de \EAS{} incorporaron una fase de reemplazo y abandonaron (al menos parcialmente) a los métodos generacionales iniciales.
%
Basado en esto, muchos autores descubrieron la posibilidad de incorporar métodos para aliviar el problema de convergencia prematura~\cite{Crepinsek:13}.
%
Es importante hacer mención de que aún cuando los métodos de reemplazamiento y generacionales~\cite{de2006evolutionary} fueron suficientemente populares, anteriormente algunos autores ya habían tomado en cuenta esta idea~\cite{mahfoud1992crowding}.
%
Sin embargo, con la efectividad del elitismo y otras estrategias de reemplazo, el número de esquemas que adoptaron estos principios crecieron de forma considerable~\cite{lozano2008replacement}.
%
\subsection{Diversidad en evolución diferencial}
%\subsection{Diversity in Differential Evolution}

Los algoritmos basados en \DE{} son altamente susceptibles a la pérdida de diversidad, esto se debe a la estrategia de selección la cual es considerada muy agresiva.
%\DE{} is highly susceptible to the loss of diversity due to the greedy strategy applied in the selection phase.
%
Sin embargo, se han desarrollado varios análisis para lidiar con este problema.
%However, several analyses to better deal with this issue have been carried out.
%
Desde que se conocen las implicaciones de cada parámetro en la diversidad, una estrategia es la estimación teórica ante distintos problemas de \DE{}~\cite{zaharie2003control}.
%Since the general implications of each parameter on the diversity are known, one of
%the alternatives is to theoretically estimate proper values for the \DE{} parameters~\cite{zaharie2003control}.
%
Alternativamente, se han desarrollado algunos análisis donde se considera el efecto de los vectores de diferencia en el operador de mutación~\cite{montgomery2009differential}.
%Differently, some analyses regarding the effects of the norm of the difference vectors used in the mutation
%have also been performed~\cite{montgomery2009differential}.
%
Estos análisis y otros estudios empíricos están basados en el operador de cruce, donde se establece un mecanismo para retrasar la convergencia prematura, particularmente éste ignora ciertos movimientos que pueden resultar perjudiciales para el rendimiento del algoritmo~\cite{montgomery2012simple}.
%Such analyses and additional empirical studies regarding the crossover allowed to conclude that some kind of movements 
%might be disallowed to delay the convergence~\cite{montgomery2012simple}.
%
En este último estudio, el tipo de movimientos aceptados varía a lo largo de la ejecución.
%In this last study the kind of accepted movements varies along the run.
%
Esto descarta movimientos menores a un umbral, el cual es decrementado conforme transcurren las generaciones.
%Specifically, it discards movements with a size below a threshold and this threshold decreases taking into account the elapsed generations.
%
Se han propuesto otras formas de alterar el procedimiento en que se aceptan los movimientos~\cite{bolufe2013differential}.
%Other ways of altering the kind of accepted movements have been proposed~\cite{bolufe2013differential}.
%
Es importante notar que este tipo de métodos tienen similitudes con nuestra propuesta en el sentido de que las decisiones están basadas por el número de generaciones transcurridas.
%Note that these kinds of methods have similarities with our proposal in the sense that decisions are biased by the number of elapsed generations.
%
%Sin embargo, nuestro método opera en la estrategia de reemplazo y no en la fase de mutación.
%However, our method operates on the replacement strategy and not on the mutation phase.
%
Mas aún, estos métodos no consideran de forma explícita los vectores de diferencias que aparecen en toda la población.
%Moreover, these methods do not consider explicitly the differences appearing on the whole population.
%
%En su lugar, las restricciones son aplicadas a las diferencias que aparecen en la fase de reemplazo.
%Instead, the restrictions apply to the differences appearing in the reproduction phase.

Una alternativa distinta reside en alterar el operador de selección~\cite{sa2008exploration}.
%A different alternative operates by altering the selection operator~\cite{sa2008exploration}.
%
Con el propósito de mantener la diversidad de la población se altera la presión de selección mediante una selección probabilística, esto permitiría escapar de las bases de atracción que pertenecen a los óptimos locales.
%Particularly, the selection pressure is relaxed through a probabilistic selection to maintain the population diversity and consequently 
%to allow escaping from basin of attraction of local optima.
%
Sin embargo este método es muy sensible al mapeo del espacio de búsqueda ya que considera la aptitud para definir las probabilidades para seleccionar a un individuo.
%Since it considers the fitness to establish the acceptance probabilities, it is very sensitive to scale transformations.
%
En este caso las decisiones no se basan en las generaciones transcurridas.
%In this case, decisions are not biased by the elapsed generations.

Finalmente, el algoritmo \textit{Diversidad de la Población Auto-Mejorado} (\textit{Auto-Enhanced Population Diversity} - \textsc{aepd}) mide la diversidad de forma explícita, así cuando se detecta poca diversidad en la población se inicia un mecanismo de diversificación~\cite{yang2015differential}.
%Finally, in the \textit{Auto-Enhanced Population Diversity} (\textsc{aepd}), the diversity is explicitly measured and it triggers a mechanism
%to diversify the population when a too low diversity is detected~\cite{yang2015differential}.
%
Es importante mencionar que ya se han propuesto estrategias con principios similares pero con esquemas de perturbación distintos~\cite{zhao2016differential}.
%Strategies with similar principles but with different disturbance schemes have also been devised~\cite{zhao2016differential}.


Particularmente, es interesante notar que las variantes de \DE{} que alcanzaron los primeros lugares en varias competencias de optimización no consideran estas modificaciones, y además estas variantes no han sido incorporadas en las herramientas de optimización más populares.
%Note that \DE{} variants with best performance in competitions do not apply these modifications
%and that most of these extensions have not been implemented in the most widely used frameworks.
%
Como resultado, estos algoritmos no son ampliamente utilizados.
%As a result, these extensions are not so widely used in the community in spite of their important benefits
%for some cases.


