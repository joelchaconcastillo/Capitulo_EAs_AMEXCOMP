Los Algoritmos Evolutivos (Evolutionary Algorithms --- \EAS{}) son considerados como uno de los enfoques con mayor eficacia para resolver distintas categorías de 
problemas de optimización.
%
Se han desarrollado diversas variantes que han sido aplicadas en múltiples campos, como en transporte, economía o ingeniería.
%
Particularmente, se han aplicado tanto en problemas del dominio continuo~\cite{glover2005handbook} como del dominio discreto~\cite{Joel:Dynamic_FAP}.
%
En general, los \EAS{} han sido especialmente exitosos en la resolución de problemas complejos en los que los enfoques exactos
no son actualmente aplicables, como por ejemplo, en problemas NP-completos con espacios de búsqueda grandes~\cite{chakraborty2008advances}.
% 
Para el ámbito de este capítulo se hace uso de problemas continuos mono-objetivo y multi-objetivo con restricciones de caja que se pueden definir 
tal y como se indica en la Ecuación (\ref{eqn:Model_general}).

\begin{equation}
 \label{eqn:Model_general}
   \begin{split}
    Minimizar \quad & \vec{F}(\vec{X})\\
%   Sujeto \quad a \quad &  x_i^{(L)} \leq x_i \leq x_i^{(U)}, \quad i=1,2,..., D \\
   Sujeto \quad a \quad & \vec{X} \in \Omega
   \end{split}
\end{equation}

donde $\vec{X}$ es un vector compuesto por $D$ variables continuas de decisión $\vec{X} = [x_1, x_2, ..., x_D]$, además cada variable pertenece al conjunto de los reales $x_i \in \Re$,
$D$ es la dimensión correspondiente al espacio de las variables de decisión, $\vec{F}$ es un vector compuesto por $M$ funciones objetivo $\vec{F} = [f_1(\vec{X}), f_2(\vec{X}), ..., f_M(\vec{X})]$,
$\Omega$ es el espacio factible cuyo límite inferior es $x_i^{(L)}$ y límite superior es $x_i^{(U)}$, es decir $\Omega = \prod _{j=1}^D[x_j^{(L)}, x_j^{(U)}]$. 
Cada vector de variables es mapeado con $M$-funciones objetivo $\vec{F}(\vec{X})( \vec{F} : \Omega \subseteq \Re^D \rightarrow \Re^M)$, al espacio $\Re^M$ que es conocido como el espacio objetivo.

Actualmente, los \EAS{} son una de las metaheurísticas más conocidas~\cite{glover2005handbook}, pero a pesar de su éxito y de su uso tan extendido, adaptar los mismos
a nuevos problemas implica la toma de varias decisiones de diseño complejas.
%
Particularmente, a la hora de diseñar de forma apropiada un \EA{}, se ha visto que es muy importante conseguir
inducir un balanceo adecuado entre la exploración e intensificación del espacio de búsqueda~\cite{herrera1996adaptation}.
%
Nótese en este punto que, de manera informal, la exploración del espacio de búsqueda consiste en evaluar regiones del espacio de búsqueda que no han sido muestreadas
con el fin de detectar regiones promisorias, y la explotación consiste en muestrear en zonas ya evaluadas previamente para realizar una búsqueda más profunda 
con el fin de encontrar soluciones más refinadas y de mayor calidad.
%
Cuando en los algoritmos evolutivos todas o casi todas las soluciones están en regiones distantes --- alta diversidad --- se produce habitualmente una búsqueda exploratoria, 
es decir, muchas de las nuevas soluciones evaluadas serán distantes a las ya evaluadas anteriormente.
%
Sin embargo, cuando casi todas las soluciones están en una o en unas pocas regiones, se produce una búsqueda intensificadora.
%
Uno de los problemas a la hora de diseñar los algoritmos evolutivos es que en muchos casos no se comprenden todas las implicaciones que los diferentes componentes
tienen sobre el mantenimiento de la diversidad de la población y por tanto, sobre el balanceo entre exploración e intensificación~\cite{Crepinsek:13}.
%
Por ello, analizar el comportamiento y rediseñar en base a lo que está ocurriendo en este aspecto, es parte del proceso de diseño de los \EAS{}.

Relacionado con lo anterior, aparece el concepto de convergencia prematura~\cite{Crepinsek:13}.
%
Se dice que un algoritmo converge de forma prematura cuando mucho antes de alcanzar el criterio de paro, todas las soluciones están en una zona muy pequeña del espacio de búsqueda.
%
En este sentido, a partir de ese momento es difícil seguir mejorando las soluciones de forma significativa ya que con alta probabilidad sólo se va a realizar un muestreo de soluciones en dicha
región.
%
Por ello, es importante detectar si esto ocurre y en tal caso rediseñar algunos aspectos del \EA{}
para preservar una mayor diversidad.
%
Sin embargo, si la población es muy diversa durante todo el proceso de búsqueda, se podría no alcanzar un grado adecuado de intensificación y por lo tanto 
se tendría una convergencia lenta que posiblemente también resultaría en soluciones de baja calidad.
%
Por esta razón, Mahfoud~\cite{dasgupta2013evolutionary} utilizó el concepto de diversidad útil para referirse a la cantidad de diversidad necesaria para generar 
soluciones de alta calidad.

En relación al diseño de \EAS{}, se puede observar que en sus inicios la mayoría de enfoques fueron esquemas generacionales~\cite{de2006evolutionary} en los 
que las soluciones hijas reemplazaban a la población anterior sin importar su respectiva aptitud o grado de diversidad.
%
En estos esquemas iniciales se usaba la selección de padres para promover que el proceso de muestreo se realizara con mayor probabilidad en las regiones más promisorias encontradas. 
%
Por ello, con el propósito de alcanzar un balanceo adecuado entre exploración e intensificación, se desarrollaron muchas estrategias de selección de padres
que permitían centrarse con mayor o menor velocidad en las regiones promisorias.
%
Además, se desarrollaron alternativas que modifican otros aspectos como la estrategia de variación~\cite{Joel:herrera2003fuzzy} y/o 
el modelo poblacional~\cite{alba2005parallel}.
%
En la mayor parte de \EAS{} más recientes, se introduce además una fase de reemplazamiento~\cite{eiben2003introduction} por lo que
la nueva población no tiene que formarse exclusivamente con los hijos,
sino que se usan mecanismos para combinar la población anterior con la población hija y determinar así los nuevos sobrevivientes.
%
En este contexto, se suele introducir elitismo en los algoritmos, es decir, el mejor individuo encontrado sobrevivirá a la siguiente generación~\cite{Joel:CHC}.
%
De esta forma, ahora también se puede modificar esta última fase para conseguir el balanceo apropiado entre exploración e intensificación.

En los últimos años, algunos de los trabajos más exitosos en el área de evitación de convergencia prematura se han basado en considerar el criterio de parada y evaluaciones
realizadas para balancear entre exploración e intensificación~\cite{segura2016novel}.
%
Esto se fundamenta en la premisa de que el grado entre exploración e intensificación debería variar a lo largo de la ejecución, por lo tanto tiene sentido que las decisiones tomadas por las componentes varíen en función del instante de ejecución.
%Esto se fundamenta en que dado que el grado entre exploración e intensificación debería variar a lo largo de la ejecución, tiene sentido que las decisiones tomadas por
%las componentes varíen en función del instante de ejecución.
%
%
Particularmente, en este capítulo se describen dos aportaciones que pertenecen a este grupo de técnicas.
%
La primera, que se aplica en el área de \textit{Evolución Diferencial} (Differential Evolution --- \DE{}), recibe el nombre de \DE{} \textit{Mejorado con Mantenimiento de Diversidad} 
(\DE{} with Enhanced Diversity Maintenance --- \DEEDM{}) e integra una estrategia de reemplazo que maneja la diversidad de forma explícita con una población élite.
%
En la fase de reemplazo que incorpora el \DEEDM{} se promueve un balanceo dinámico entre exploración e intensificación, teniendo en cuenta para ello el criterio de paro
y las evaluaciones transcurridas para la toma de decisiones.
%
Concretamente, en las primeras etapas se induce un grado de exploración alto ya que los individuos sobrevivientes son diversificados, y posteriormente conforme transcurren las 
generaciones se induce un mayor grado de intensificación.
% 
La segunda aportación está enfocada a la extensión de operadores de cruce.
%
Particularmente se analizan los componentes que conforman al \textit{Operador de Cruce basado en Simulación Binaria} (Simulated Binary Crossover - SBX), 
y se propone una variante dinámica (Dynamic Simulated Binary Crossover --- \DSBX{}) en la que el comportamiento interno es alterado en base al criterio de paro y evaluaciones transcurridas.

El resto de este capítulo está organizado de la siguiente forma.
%
Inicialmente, en la sección \ref{sec:Mecanismos} se revisan algunas de las técnicas más populares que se han propuesto para promover el mantenimiento de diversidad, exponiendo una de las clasificaciones más extendidas.
%
Posteriormente, en la sección \ref{sec:ED} se describe y valida experimentalmente la aportación basada en evolución diferencial donde se considera la diversidad de forma explícita en la fase de reemplazamiento.
%
Siguiendo esta misma línea en la sección \ref{sec:Operadores} se lleva a cabo un análisis del operador de cruce \SBX{} mencionando algunas de sus implicaciones en la diversidad, 
y se expone en base a ello una variante dinámica que se valida experimentalmente con problemas multi-objetivo.
%
Finalmente, en la sección \ref{sec:Conclusion} se exponen las conclusiones y trabajos futuros trazados en base a los resultados obtenidos.
%
