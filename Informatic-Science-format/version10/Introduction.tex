Los Algoritmos Evolutivos (Evolutionary Algorithms - \EAS{}) son considerados como uno de los enfoques con mayor eficacia para resolver distintas categorías de 
problemas de optimización.
%
Se han desarrollado diversas variantes que han sido aplicadas en múltiples campos, como es en transporte, economía o ingeniería.
%
Particularmente, se han aplicado tanto en problemas del dominio continuo~\cite{glover2005handbook} como en el dominio discreto~\cite{Joel:Dynamic_FAP}.
%
En general, los \EAS{} han sido especialmente exitosos en la resolución de problemas complejos en los que los enfoque exactos
no han podido ser usados, como por ejemplo, en problemas NP-completos con espacios de búsqueda grandes~\cite{chakraborty2008advances}.
% 
Para el ámbito de este artículos, se puede definir un problema de optimización como se indica en la ecuación (\ref{eqn:Model_general}).

\begin{equation}
 \label{eqn:Model_general}
   \begin{split}
    Minimizar \quad & f_m(\vec{x}), \quad m = 1, 2,...,M;\\
   Sujeto \quad a \quad &  x_i^{(L)} \leq x_i \leq x_i^{(U)}, \quad i=1,2,..., D \\
   & \mathbf{x} \in \Omega
   \end{split}
\end{equation}

donde $\vec{X}$ es un vector compuesto por $D$ variables de decisión $\vec{X} = [x_1, x_2, ..., x_D]$, $D$ es la dimensión correspondiente al espacio de las variables de decisión, $\Omega$ es el espacio factible cuyo límite inferior es $x_i^{(L)}$ y límite superior es $x_i^{(U)}$, y en la que cada solución es evaluada mediante el uso de la función $F : \Omega \rightarrow R^M$, la cual consiste de $M$ funciones objetivo, siendo $R^M$ conocido como el espacio objetivo.

Actualmente, los \EAS{} son una de las metaheurísticas más conocidas~\cite{glover2005handbook}, y a pesar de su éxito y uso muy extendido, adaptar los mismos
a nuevos problemas implican la toma de varias decisiones de diseño complejas.
%
Particularmente, a la hora de diseñar de forma apropiada un \EA{}, se ha visto que es muy importante conseguir
inducir un balance adecuado entre la exploración e intensificación del espacio de búsqueda~\cite{herrera1996adaptation}.
%
Nótese en este punto que, de manera informal, la exploración del espacio de búsqueda consiste en detectar regiones del espacio de búsqueda que contiene soluciones
promisorias, y la explotación consiste en dada una de esas regiones, realizar una búsqueda más profunda para encontrar soluciones más refinadas.
%
Cuando en los algoritmos evolutivos todas o casi todas las soluciones se encuentran en diferentes regiones, se produce una búsqueda explorativa, mientras que cuando 
casi todas las soluciones están en una o en unas pocas regiones, se produce una búsqueda intensificadora.
%
Uno de los problemas a la hora de diseñar los algoritmos evolutivos, es que en muchos casos, no se comprenden todas las implicaciones que los diferentes componentes
tienen sobre el proceso de mantener la diversidad y sobre el balanceo entre exploración e intensificación~\cite{Crepinsek:13}, por lo que analizar y rediseñar en base
a lo que está ocurriendo en este aspecto, es parte del proceso de diseño de los algoritmos evolutivos.

Relacionado con lo anterior, aparece el concepto de convergencia prematura~\cite{Crepinsek:13}.
%
Se dice que un algoritmo converge de forma prematura cuando antes de alcanzar el criterio de paro, todas las soluciones están en una zona muy próxima.
%
En este sentido, a partir de ese momento es muy difícil seguir mejorando las soluciones, por lo que si esto ocurre se deben rediseñar algunos aspectos
para preservar una mayor diversidad.
%
Sin embargo, si la población es muy diversa durante todo el proceso de búsqueda, no se podría alcanzar un grado adecuado de intensificación y por lo tanto 
se tendría una convergencia lenta que posiblemente también resultaría en soluciones de baja calidad.
%
Por esta razón, Mahfoud~\cite{dasgupta2013evolutionary} utilizó el concepto de diversidad útil para referirse a la cantidad de diversidad necesaria para generar 
soluciones de alta calidad.

En relación al diseño de los \EAS{}, se puede observar que en sus inicios, la mayoría de enfoques fueron esquemas generacionales~\cite{de2006evolutionary} en los 
que las soluciones hijas reemplazaban a la población anterior sin importar su respectiva aptitud o grado de diversidad.
%
En estos esquemas iniciales se usaba la selección de padres para mover el proceso de búsqueda hacia las regiones más prometedoras.
%
En estas estrategias, con el propósito de alcanzar un balance adecuado entre exploración e intensificación, se desarrollaron muchas estrategias de selección de padres
que permitían centrarse con mayor o menor velocidad en las mejores zonas encontradas.
%
Además, se desarrollaron otras alternativas en las cuales se modificaba la estrategia de variación~\cite{Joel:herrera2003fuzzy} y/o el modelo poblacional~\cite{alba2005parallel}.
%
En la mayor parte de \EAS{} más recientes, se introduce además la fase de reemplazamiento~\cite{eiben2003introduction}, es decir, la nueva población no tiene que formarse exclusivamente con los hijos,
sino que se usan mecanismos para combinar la población anterior con la población hija y determinar así los nuevos sobrevivientes.
%
En este contexto, se suele introducir elitismo en los algoritmos, es decir, el mejor individuo encontrado sobrevivirá a la siguiente generación~\cite{Joel:CHC}.
%
De esta forma, ahora también se puede modificar esta última fase para conseguir el balanceo apropiado entre exploración e intensificación.

En los últimos años, algunos de los trabajos más exitosos en el área de evitación de convergencia prematura se han basado en considerar el criterio de parada y evaluaciones
realizadas para balancear entre exploración e intensificación~\cite{segura2016novel}.
%
Esto se fundamenta en las fases de variación y selección de padres, estas fases realizan decisiones que podrían afectar a las generaciones actuales ya que la elección de los sobrevivientes influye  de forma significativa a todo proceso de optimización.
%
Especialmente, este mecanismo se enfoca en seleccionar a las soluciones que sobreviven en la siguiente generación, por lo tanto la fase de reemplazo puede actuar adecuadamente evitando la selección de los individuos no deseados que provienen de las fases de variación y selección de padres.

%
Particularmente, en este capítulo se describen dos aportaciones, la primera está enfocada al área de \textit{Evolución Diferencial} (Differential Evolution \DE{}),esta primera aportación es nombrada \DE{} \textit{Mejorado con Mantenimiento de Diversidad} (\DE with Enhanced Diversity Maintenance \DEEDM{}) e integra una estrategia de reemplazo que mantiene un grado de diversidad de forma explícita considerando además una población elite.
%
En la fase de reemplazo que incorpara el \DEEDM{} se promueve un balance entre exploración e intensificación de forma dinámica donde es considerando el criterio de paro.
%
De esta forma, en las primeras etapas se induce un grado de exploración ya que los individuos sobrevivientes son diversificados, posteriormente conforme transcurren las generaciones y de forma gradual se induce un grado de intecificación.
% 
La segunda aportación están enfocada a los operadores de cruce, particularmente se analizan los componentes que conforman a \textit{El Operador de Cruce basado en Simulación Binaria} (Simulated Binary Crossover - SBX), y además se propone una variante \DSBX{} el cual es considerado como una modificación del operador \SBX{} cuyo comportamiento interno es alterado de forma dinámica con el propósito de alcanzar un balance.
%
El resto de este capítulo está organizado de la siguiente forma.
