From the experimental results in this paper, several conclusions can be drawn.
%

First, from experimental research on the working mechanism, it can be seen that our proposal is able to relieve the premature convergence to several optimization levels.
%
Second, our proposal is able to enhance the performance of \DE{} algorithms, in particular when the search space is large.
%
Third, it is also less sensitive to the parameter of population size, so our proposal can be competitive even if the population size is small.
%
Fourth, it seems that our proposal has some drawbacks in relation with the proportion of difference vectors, due that the displacement of the vectors is directly related with the diversity promoted in the population.
%
Thus, in some cases the minimum displacement bounded by the diversity that is explicitly promoted in the population.


For future work of this paper, two interesting issues should be addressed for our proposal.
%
The first one is that explored areas in the search space should be avoided to save computing resources.
%
Development an adaptive strategy for the distance factor should involve a more stable algorithm.
%
Explore the possibility to involve a local search scheme with two goals, save the function evaluations and consider irregular fitness landscapes (e.g. multi-modal problems).
%
Applying our proposal to real-world problems should be an interesting topic.
%
%Based in several analyzes the mutation factor could be selected inside the distance factor, then develop a strategy where this parameter is no required.
%
Generate a theoretical model to select the adequately population size given a initial distance factor.
%
Finally, implement the replacement procedure to the Estimation of Distribution Algorithms seems a promising field.
